%------------------------------------------------------------------------------
% Copyright (c) !!COPYRIGHTYEAR!!, Xavier Leroy and Didier Remy.  
%
% All rights reserved. Distributed under a creative commons
% attribution-non-commercial-share alike 2.0 France license.
% http://creativecommons.org/licenses/by-nc-sa/2.0/fr/
%
% Translation by
%------------------------------------------------------------------------------

\begin{titlingpage}
%% Title page
\maketitle
\newpage

%% Copyright page
\begin{copyrightnotice}
\textcopyright{} 1991, 1992, 2003, 2004, 2005, 2006, 2008, 2009 \\
\myauthors, \textsc{inria} Rocquencourt.\\
Droits r�serv�s.
\ifhtmlelse
    {Voir les \href{LICENSE}{termes l�gaux.}  \href{\licenseURL}%
      {\imgsrc[alt="CreativeCommons License" class="ccimage"]%
        {http://i.creativecommons.org/l/by-nc-sa/2.0/fr/80x15.png}}
    }
    {Distribu� sous licence Creative Commons Paternit�~--~Pas
      d'Utilisation Commerciale~--~Partage des Conditions Initiales �
      l'Identique 2.0 France. Voir \url{\licenseURL} pour les termes
      l�gaux.}
\end{copyrightnotice}

\ifhtml{ 
Disponible en version \ahref{ocamlunix.html}{monolitique},
\ahref{index.html}{par chapitres}, et en \ahref{ocamlunix.pdf}{PDF}
--- les \ahref{ocamlunix-!!VERSION!!.tbz}{sources}.} 
\vfill
\begin{abstract}
Ce document est un cours d'introduction � la programmation du syst�me
Unix, mettant l'accent sur la communication entre les processus. La
principale nouveaut� de ce travail est l'utilisation du langage
Objective Caml, un dialecte du langage ML, � la place du langage C qui
est d'ordinaire associ� � la programmation syst�me. Ceci donne des
points de vue nouveaux � la fois sur la programmation syst�me et sur
le langage ML.
\end{abstract}
\end{titlingpage}

%% Table of contents
\ifhtmlelse{\tableofcontents\cutname{toc.html}}{\tableofcontents*}
\newpage
